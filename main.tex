\PassOptionsToPackage{dvipsnames}{xcolor}
\documentclass[10pt,a4paper]{altacv}

%Layout
\geometry{marginparwidth=6.8cm,marginparsep=1.2cm,footskip=2\baselineskip}

%Packages
\usepackage[utf8]{inputenc}
\usepackage[T1]{fontenc}
\usepackage[default]{lato}
\usepackage{hyperref}

%Colors

\definecolor{accent}{HTML}{000E17}
\definecolor{heading}{HTML}{000E17}
\definecolor{emphasis}{HTML}{696969}
\definecolor{body}{HTML}{01415F}
% \definecolor{body}{HTML}{110000}

\colorlet{heading}{heading}
\colorlet{accent}{accent}
\colorlet{emphasis}{emphasis}
\colorlet{body}{body}

\renewcommand{\itemmarker}{{\small\textbullet}}
\renewcommand{\ratingmarker}{\faCircle}

%

\begin{document}
\name{Simone Bologna}
\tagline{\linkedin{smnbologna} \github{simonecid}}
%\photo{2.5cm}{BrunoAlves}
\personalinfo{
    \email{smnbologna@gmail.com}
    \phone{+39 320 110 34 66}
    \printinfo{\faBirthdayCake}{27th July 1992}
    \location{Vicolo Plinio 14, Valmorea, Italy}
    % \linkedin{linkedin.com/in/smnbologna}
    % \github{github.com/simonecid}
}

%

\begin{fullwidth}
\makecvheader
%
\end{fullwidth}

%%%%%%%%%%%%%%%%%%%%%%%%%%%%%%% Experience %%%%%%%%%%%%%%%%%%%%%%%%%%%%%%%

\cvsection{Qualifications}

%%%%%%%%%%%%%
%Experience 1
%%%%%%%%%%%%%

\cvevent{PhD}{University of Bristol \& CERN}{Expected end of 2020}{Bristol, UK \& Geneva, Switzerland}

\begin{itemize}
    \setlength{\itemindent}{0.5em}
    % \item[--]   \small{3.5-year course, 2 years in Bristol and 1.5 years at CERN}
    \item[--]   \small{\underline{Expected award date: end of 2020}}
    \item[--]   \small{Thesis draft title: \textit{Triggering at high-luminosity hadron colliders}}
    \item[--]   \small{Supervisors: Dr. Jim Brooke and Prof. Joel Goldstein}
\end{itemize}

\medskip

\cvevent{Master Degree in Particle Physics}{Università degli Studi di Milano – Bicocca}{2016}{Milan, Italy}

\begin{itemize}
    \setlength{\itemindent}{0.5em}
    \item[--]   \small{\textbf{Final mark: 110/110 cum laude}}
    \item[--]   \small{Thesis title: \textit{Upgrade of the CMS Level-1 Trigger and efficiency study of the tau}}
    \item[--]   \small{Supervisors: Dr. Alessandro Thea and Prof. Marco Paganoni} 
\end{itemize}

\medskip

\cvevent{Bachelor in Physics}{Università degli Studi di Milano – Bicocca}{2014}{Milan, Italy}

\begin{itemize}
    \setlength{\itemindent}{0.5em}
    \item[--]   \small{\textbf{Final mark: 110/110 cum laude}}
    \item[--]   \small{Thesis title: \textit{A cosmic-ray stand to test the response of
microchannel-plate detectors to ionizing particles}}
    \item[--]   \small{Supervisors: Prof. Tommaso Tabarelli de Fatis} 
\end{itemize}

\cvsection{Awarded fellowships and studentships}

\cvsection{Research experience}

\cvevent{Doctoral researcher}{University of Bristol \& CERN}{February 2017 -- Ongoing}{Bristol, UK \& Geneva, Switzerland}

\begin{itemize}
    \setlength{\itemindent}{0.5em}
    \item[--]   \small{I was based for two years in Bristol and for one year and half at CERN}
    \item[--]   \small{I built a \textbf{parameterised simulation of the level-1 trigger} of the CMS experiment to estimate trigger rates of a system with equivalent performance running at the Future Circular Collider (FCC) and \textbf{understand challenges of data acquisition systems at FCC.} 
    %I first computed parameters modelling the performance of jet, muon, and e/$\gamma$ triggers. Then, I compared the results of the parameterised simulation with the ones from the CMS full-simulation for validation. Finally, I ran the parameterised simulation on events at the FCC-hh energy and luminosity. 
    The results of my work showed that detectors at FCC-hh may have to develop sophisticated multi-object triggers in order to be sensitive to interesting physics in the high pile-up environment of the experiment and maintain a reasonable accept rate at the same time. 
    Results from this work contributed to the FCC-hh Conceptual Design Report [1] and to defining the \textbf{European Strategy for Particle Physics}}.
    \item[--]   \small{I \textbf{developed the flagship algorithm for detecting hadron jets and computing energy sums} of the Phase-2 upgrade of level-1 trigger in the CMS detector. I first designed the algorithm and studied its reconstruction performance in Geant4 simulations of the CMS detector to verify that it provided satisfactory reconstruction performance. I, then, \textbf{wrote firmware targeting Xilinx KU115, KU15P and VU9P FPGAs using High-Level Synthesis} tools. I run the algorithm on hardware mounting KU115 and KU15P FPGAs and validated the output against CMS simulations obtaining an agreement rate of 96\%. 
    Results from this work were included in the technical design report of the Phase-2 upgrade of the CMS level-1 trigger [2].}
    \item[--] I contributed to \textbf{maintaining the Online Software of the level-1 trigger of the CMS experiment}. This work was fundamental in ensuring smooth detector operation during the Run-2 of the CMS experiment by providing information on the system status to experts and helping in diagnosing issues occurring during data taking.
    \item[--] I was the \textbf{on-call expert of the level-1 trigger of the CMS experiment} for three weeks. I was responsible for preparing the system for data taking and solving any potential issue that could arise during operations.
\end{itemize}

\medskip

\cvevent{INFN Fellowship}{CERN}{November 2016 -- January 2017}{Geneva, Switzerland}

\begin{itemize}
    \setlength{\itemindent}{0.5em}
    \item[--]   \small{I developed an \textbf{algorithm to find hit doublets in the CMS tracker} for \textbf{Nvidia GPU}s using \textbf{CUDA}. This work was performed in view of the Run-3 of CMS when GPUs will be employed in high-level trigger to reconstruct tracks.}
\end{itemize}

\medskip

\cvevent{Master Degree research project}{Università degli Studi di Milano – Bicocca}{January - September 2016}{Milan, Italy \& Geneva, Switzerland}

\begin{itemize}
    \setlength{\itemindent}{0.5em}
    \item[--]   I was based for six months at CERN.
    \item[--]   \small{I computed the $\tau$ \textbf{trigger efficiency turn-on curve} by applying tag and probe techniques to Z$^0\to\tau\tau$ decays. My work contributed to the $\mathrm{HH} \to \mathrm{bb}\tau\tau$ analysis of the CMS experiment, performed on data collected in 2016. Results from this analysis was presented at the 2016 ICHEP conference (??? CHIEDI ???).}
    \item[--]   \small{I \textbf{developed the general control and monitoring web interface} for the CMS L1-Trigger. This interface provides detailed information on the system status and it was routinely used by experts during Run-2 to promptly diagnose critical issues occurring during data taking.}
\end{itemize}

\medskip

\cvevent{Bachelor Degree research project}{Università degli Studi di Milano – Bicocca}{July - August 2014}{Milan, Italy}

\begin{itemize}
    \setlength{\itemindent}{0.5em}
    \item[--] I measured the efficiency and time resolution of micro-channel plate (MCP) detectors. I measured a time resolution of 100 ps and an efficiency of 50\% in a non-optimized detector. This preliminary work was the first step of the R\&D aimed to investigate whether the MCP could be exploited for high-resolution timing applications in the CMS detector.
\end{itemize}

\cvsection{Teaching experience}

%%%%%%%%%%%%%
%Experience 2
%%%%%%%%%%%%%

\cvevent{Tutor}{International School of Trigger and Data Acquisition}{13th -- 22nd January 2020}{Valencia, Spain}

\begin{itemize}
    \setlength{\itemindent}{0.5em}
    \item[--]   \small{I tutored a lab on Arduino microcontrollers introducing students to basics of programming these devices.}
    \item[--]   MOVE ME: I got called after performing well in the laboratory as a student.
\end{itemize}

\medskip

\cvevent{Demonstrator}{University of Bristol}{February -- April 2017 \& 2019}{Bristol, UK}

\begin{itemize}
    \setlength{\itemindent}{0.5em}
    \item[--]   \small{for computational physics course What did I do? Marking.}
\end{itemize}

\cvsection{Attended schools}

\cvevent{}{International School of Trigger and Data Acquisition}{3rd -- 12th April 2019}{Egham, UK}

\begin{itemize}
    \setlength{\itemindent}{0.5em}
    \item[--]   \small{Introductory school to triggering and acquiring data for physics experiments.}
\end{itemize}

\medskip

\cvevent{}{CERN School of Computing}{13th -- 26th September 2015}{Geneva, Switzerland}

\begin{itemize}
    \setlength{\itemindent}{0.5em}
    \item[--]   \small{Course on Scientific Computing for High Energy Physics.}
\end{itemize}

\medskip


%%%%%%%%%%%%%%%%%%%%%%%%%%%%%%% Projects %%%%%%%%%%%%%%%%%%%%%%%%%%%%%%%

% \cvsection[page2sidebar]{Publications and talks}

% \begin{enumerate}
%   \item \small{FCC collaboration, Future Circular Collider Study. Volume 3: The Hadron Collider (FCC-hh) Conceptual Design Report, CERN-ACC-2018-0058, Geneva, December 2018. Published in Eur. Phys. J. ST., p. 973-975}
%   \item \small{S. Bologna et al. “Trigger \& Data Acquisition at FCC-hh”. Talk at the FCC Week 2017, Berlin. 29th May - 2nd June 2017. 
% URL: \url{https://indico.cern.ch/event/556692/contributions/2465162/}}
%   \item \small{S. Bologna et al. “Trigger \& Data Acquisition at FCC-hh”. Talk at the FCC Week 2018, Amsterdam. 12th April 2018. 
% URL: \url{https://indico.cern.ch/event/656491/contributions/2923411/}}
% \item \small{CMS Collaboration, The Phase-2 Upgrade of the CMS Level-1 Trigger, Tech. Rep. 1274, CERN-LHCC-2020-004. CMS-TDR-021, CERN, Geneva, Apr 2020.}
% ICHEP
% \end{enumerate}


%%%%%%%%%%%%%%%%%%%%%%%%%%%%%%% Certificates %%%%%%%%%%%%%%%%%%%%%%%%%%%%%%%

% \cvsection{Certificates}

%%%%%%%%%%%%%
%Certificate 1
%%%%%%%%%%%%%

% \cvevent{MCSD: App Builder — Certified 2019}{Microsoft}{Jan 2019 -- No Expiration Date}{}
% \divider

%%%%%%%%%%%%%
%Certificate 2
%%%%%%%%%%%%%

% \cvevent{MCSA: Web Applications — Certified 2019}{Microsoft}{Jan 2019 -- No Expiration Date}{}
% \divider

%%%%%%%%%%%%%%%%%%%%%%%%%%%%%%% Interests %%%%%%%%%%%%%%%%%%%%%%%%%%%%%%%

% \cvsection{Interests}

% \wheelchart{1.5cm}{0.5cm}{%
%   10/10em/accent!30/Electronics,
%   25/9em/accent!60/Science,
%   5/13em/accent!10/Movies,
%   20/15em/accent!40/Sport,
%   30/9em/accent/Technology,
%   5/8em/accent!20/Literature
% }


%%New Page
% \clearpage



\end{document}

