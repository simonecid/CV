\PassOptionsToPackage{dvipsnames}{xcolor}
\documentclass[10pt,a4paper]{altacv}

%Layout
\geometry{left=1cm,right=9cm,marginparwidth=6.8cm,marginparsep=1.2cm,top=1.25cm,bottom=1.25cm,footskip=2\baselineskip}

%Packages
\usepackage[utf8]{inputenc}
\usepackage[T1]{fontenc}
\usepackage[default]{lato}
\usepackage{hyperref}

%Colors

\definecolor{accent}{HTML}{000e17}
\definecolor{heading}{HTML}{000e17}
\definecolor{emphasis}{HTML}{696969}
\definecolor{body}{HTML}{01415f}
% \definecolor{body}{HTML}{110000}

\colorlet{heading}{heading}
\colorlet{accent}{accent}
\colorlet{emphasis}{emphasis}
\colorlet{body}{body}

\renewcommand{\itemmarker}{{\small\textbullet}}
\renewcommand{\ratingmarker}{\faCircle}

%

\begin{document}
\name{Simone Bologna}
\tagline{\linkedin{smnbologna} \github{simonecid}}
%\photo{2.5cm}{BrunoAlves}
\personalinfo{
    \phone{+39 320 110 34 66}
    \email{smnbologna@gmail.com}
    \printinfo{\faBirthdayCake}{27th July 1992}
    \location{Vicolo Plinio 14, Valmorea, Italy}
%    \mailaddress{SR Njemačke 6, 10000}
    % \linkedin{linkedin.com/in/smnbologna}
    % \github{github.com/simonecid}
}

%

\begin{fullwidth}
\makecvheader
\end{fullwidth}

%

%%%%%%%%%%%%%%%%%%%%%%%%%%%%%%% Experience %%%%%%%%%%%%%%%%%%%%%%%%%%%%%%%

\cvsection[page1sidebar]{Education}

%%%%%%%%%%%%%
%Experience 1
%%%%%%%%%%%%%

\cvevent{PhD student}{University of Bristol \& CERN}{February 2017 -- Ongoing}{Bristol, UK \& Geneva, Switzerland}

\begin{itemize}
    \setlength{\itemindent}{0.5em}
    \item[--]   \small{3.5-year course, 2 years in Bristol and 1.5 years at CERN}
    \item[--]   \small{\underline{Expected thesis submission date: September 2020}}
    \item[--]   \small{I built a \textbf{parameterised simulation of the level-1 trigger} of the CMS experiment to estimate trigger rates of a system with equivalent performance running at the Future Circular Collider (FCC). I used the results to \textbf{understand challenges of data acquisition systems at FCC.}}
    % \underline{Relevant publications and talks: [1], [2], [3].}}
    \item[--]   \small{I \textbf{developed the flagship algorithm for detecting hadron jets} of the Phase-2 upgrade of level-1 trigger in the CMS detector. I studied its physics performance in simulations, and I \textbf{wrote and tested firmware targeting Xilinx FPGAs using High-Level Synthesis} tools. 
    Results will be published later this year.}
    % \underline{Relevant publications and talks: [4].}}
\end{itemize}

\medskip

%%%%%%%%%%%%%
%Experience 2
%%%%%%%%%%%%%

\cvevent{Tutor at international school}{International School of Trigger and Data Acquisition}{13th -- 22nd January 2020}{Valencia, Spain}

%\begin{itemize}
%    \item Fullstack JAVA developer
%\end{itemize}

\begin{itemize}
    \setlength{\itemindent}{0.5em}
    \item[--]   \small{Tutored a lab on the Arduino microcontroller}
\end{itemize}

\medskip

\cvevent{Student at international school}{International School of Trigger and Data Acquisition}{3rd -- 12th April 2019}{Egham, UK}

\begin{itemize}
    \setlength{\itemindent}{0.5em}
    \item[--]   \small{Introductory school to triggering and acquiring data for physics experiments}
\end{itemize}

\medskip

\cvevent{Fellowship}{CERN}{November 2016 -- January 2017}{Geneva, Switzerland}

\begin{itemize}
    \setlength{\itemindent}{0.5em}
    \item[--]   \small{Developed high-performance software for \textbf{Nvidia GPU}s in \textbf{CUDA} for the high-level trigger of the CMS experiment}
\end{itemize}

\medskip

\cvevent{Student at international school}{CERN School of Computing}{13th -- 26th September 2015}{Geneva, Switzerland}

\begin{itemize}
    \setlength{\itemindent}{0.5em}
    \item[--]   \small{Course on Scientific Computing for High Energy Physics}
\end{itemize}

\medskip

\cvevent{Bachelor \& Master Degree in Particle Physics}{Università degli Studi di Milano – Bicocca}{2011 -- 2016}{Milan, Italy}

\begin{itemize}
    \setlength{\itemindent}{0.5em}
    \item[--]   \small{Master thesis title: \textit{Upgrade of the CMS Level-1 Trigger and efficiency study of the tau}}:
    \begin{itemize}
    \item[--]   \small{I computed the $\tau$ \textbf{trigger efficiency turn-on curve} by applying tag and probe techniques to Z$^0\to\tau\tau$ decays. }
    \item[--]   \small{I \textbf{developed the general control and monitoring web interface} for the CMS L1-Trigger. }
    \end{itemize}
    % This interface provides detailed information on the system status and it is routinely used by experts to promptly diagnose critical issues occurring during data taking.
\end{itemize}

\cvevent{High School Degree in Information Technology}{Istituto Tecnico Industriale Statale Magistri Cumacini (Technical Institute)}{2006 -- 2011}{Como, Italy}~\\[4cm]

%%%%%%%%%%%%%%%%%%%%%%%%%%%%%%% Projects %%%%%%%%%%%%%%%%%%%%%%%%%%%%%%%

% \cvsection[page2sidebar]{Publications and talks}

% \begin{enumerate}
%   \item \small{FCC collaboration, Future Circular Collider Study. Volume 3: The Hadron Collider (FCC-hh) Conceptual Design Report, CERN-ACC-2018-0058, Geneva, December 2018. Published in Eur. Phys. J. ST., p. 973-975}
%   \item \small{S. Bologna et al. “Trigger \& Data Acquisition at FCC-hh”. Talk at the FCC Week 2017, Berlin. 29th May - 2nd June 2017. 
% URL: \url{https://indico.cern.ch/event/556692/contributions/2465162/}}
%   \item \small{S. Bologna et al. “Trigger \& Data Acquisition at FCC-hh”. Talk at the FCC Week 2018, Amsterdam. 12th April 2018. 
% URL: \url{https://indico.cern.ch/event/656491/contributions/2923411/}}
% \item \small{CMS Collaboration, The Phase-2 Upgrade of the CMS Level-1 Trigger, Tech. Rep. 1274, CERN-LHCC-2020-004. CMS-TDR-021, CERN, Geneva, Apr 2020.}
% \end{enumerate}


%%%%%%%%%%%%%%%%%%%%%%%%%%%%%%% Certificates %%%%%%%%%%%%%%%%%%%%%%%%%%%%%%%

% \cvsection{Certificates}

%%%%%%%%%%%%%
%Certificate 1
%%%%%%%%%%%%%

% \cvevent{MCSD: App Builder — Certified 2019}{Microsoft}{Jan 2019 -- No Expiration Date}{}
% \divider

%%%%%%%%%%%%%
%Certificate 2
%%%%%%%%%%%%%

% \cvevent{MCSA: Web Applications — Certified 2019}{Microsoft}{Jan 2019 -- No Expiration Date}{}
% \divider

%%%%%%%%%%%%%%%%%%%%%%%%%%%%%%% Interests %%%%%%%%%%%%%%%%%%%%%%%%%%%%%%%

% \cvsection{Interests}

% \wheelchart{1.5cm}{0.5cm}{%
%   10/10em/accent!30/Electronics,
%   25/9em/accent!60/Science,
%   5/13em/accent!10/Movies,
%   20/15em/accent!40/Sport,
%   30/9em/accent/Technology,
%   5/8em/accent!20/Literature
% }


%%New Page
% \clearpage




\end{document}
